
\documentclass[]{article}
\usepackage[landscape]{geometry}
\usepackage{url}
\usepackage{multicol}
\usepackage{amsmath}
\usepackage{esint}
\usepackage{bigints}
\usepackage{amsfonts}
\usepackage{xcolor}
\usepackage{tikz}
\usetikzlibrary{calc}
\usetikzlibrary{decorations.pathmorphing}
\usepackage{amsmath,amssymb}
\usepackage{tabularx}
\usepackage{colortbl}
\usepackage{xcolor}
\usepackage{mathtools}
\usepackage{amsmath,amssymb}
\usepackage{enumitem}
\usepackage{xhfill}
\makeatletter
\usepackage{listings}
\usepackage{color}

\definecolor{dkgreen}{rgb}{0,0.6,0}
\definecolor{gray}{rgb}{0.5,0.5,0.5}
\definecolor{mauve}{rgb}{0.58,0,0.82}

\lstset{frame=tb,
  language=C,
  aboveskip=3mm,
  belowskip=3mm,
  showstringspaces=false,
  columns=flexible,
  basicstyle={\small\ttfamily},
  numbers=none,
  numberstyle=\tiny\color{gray},
  keywordstyle=\color{blue},
  commentstyle=\color{dkgreen},
  stringstyle=\color{mauve},
  breaklines=true,
  breakatwhitespace=true,
  tabsize=3
}
\newcommand*\bigcdot{\mathpalette\bigcdot@{.5}}
\newcommand*\bigcdot@[2]{\mathbin{\vcenter{\hbox{\scalebox{#2}{$\m@th#1\bullet$}}}}}
\makeatother

\title{CAB202 Exam Cheat Sheet}
\usepackage[utf8]{inputenc}
\advance\topmargin-.8in
\advance\textheight3in
\advance\textwidth3in
\advance\oddsidemargin-1.5in
\advance\evensidemargin-1.5in
\parindent0pt
\parskip2pt
\newcommand{\hr}{\centerline{\rule{3.5in}{1pt}}}
\newcommand{\nc}[2][]{%
\tikz \draw [draw=black, ultra thick, #1]
    ($(current page.center)-(0.5\linewidth,0)$) -- 
    ($(current page.center)+(0.5\linewidth,0)$)
    node [midway, fill=white] {#2};
}
\begin{document}

\begin{center}{\huge{\textbf{CAB202 Exam Cheat Sheet}}}\\
\end{center}
\begin{multicols*}{3}

\tikzstyle{mybox} = [draw=black, fill=white, very thick,
    rectangle, rounded corners, inner sep=10pt, inner ysep=10pt]
\tikzstyle{fancytitle} =[fill=black, text=white, font=\bfseries]

%--------------------------
\begin{tikzpicture}
\node [mybox] (box){%
    \begin{minipage}{0.3\textwidth}
\addtolength{\itemsep}{-2pt}
%\item[$g = $]  
%  9.80 m s$^{-2}$
    - Compilation adds complexity\\
    - Few data structures, small standard library\\
    - Typically more code required for a given application\\
    - Not memory-safe and only weakly-typed; easy to write buggy code (eg.
buffer overflow)
    \end{minipage}
};
%---------------------------------
\node[fancytitle, right=10pt] at (box.north west) {Downsides of C};
\end{tikzpicture}

%---------------------------
\begin{tikzpicture}
\node [mybox] (box){%
    \begin{minipage}{0.3\textwidth}
    \begin{itemize}
\addtolength{\itemsep}{-2pt}
    \begin{tabularx}{0.75\textwidth} { 
  | >{\raggedright\arraybackslash}X 
  | >{\centering\arraybackslash}X 
  | >{\raggedleft\arraybackslash}X | }
     \hline
     Operator & Description & Example \\
     \hline
     ==&Equal&1 == 1\\
     \hline
     !=&Not Equal&1 != 0\\
     \hline
     $>$&Greater&$3>2$\\
    \hline
    $<$&Less&$3<4$\\
    \end{tabularx}
\end{itemize}
\end{minipage}
};
%---------------------------------
\node[fancytitle, right=10pt] at (box.north west) {Rational Operators};
\end{tikzpicture}

%---------------------------
\begin{tikzpicture}
\node [mybox] (box){%
    \begin{minipage}{0.3\textwidth}
    - Unlike other languages, C doesn’t have Boolean data type.\\
    - Any non-zero value is true, 0 is false\\
    - Be sure to use == to test equality, = for assignment.\\
    - Identation
    \end{minipage}
};
%---------------------------------
\node[fancytitle, right=10pt] at (box.north west) {Common Problems};
\end{tikzpicture}

%---------------------------
\begin{tikzpicture}
\node [mybox] (box){%
    \begin{minipage}{0.3\textwidth}
    \begin{tabularx}{1.0\textwidth} { 
  | >{\raggedright\arraybackslash}X 
  | >{\centering\arraybackslash}X | }
     \hline
     Operator & Description \\
     \hline
     \&\& & AND\\
     \hline
     \textbar \hspace{} \textbar&OR\\
     \hline
     \&&BITWISE AND\\
    \hline
    \textbar&BITWISE OR\\
    \hline
    \^{}&BITWISE XOR\\
    \hline
    \&=&AND EQUAL\\
    \hline
    \textbar=&OR EQUAL\\
    \hline
    \^{}=&XOR EQUAL\\
    \end{tabularx}
    \end{minipage}
};
%------------ campo titulo---------------------
\node[fancytitle, right=10pt] at (box.north west) {Logical Expressions};
\end{tikzpicture}
%---------------------------------
%---------------------------
\begin{tikzpicture}
\node [mybox] (box){%
    \begin{minipage}{0.3\textwidth}
    - Can stop a loop early by using the break statement
    \begin{lstlisting}
    int i = 1000;
    while ( i < 2000) {
    // Stop when we find a multiple of 87
    if ( i % 87 == 0) { break ; }
    i ++;
    }
    printf ( " % d \ n " , i );
    \end{lstlisting}
    \end{minipage}
};
%---------------------------------
\node[fancytitle, right=10pt] at (box.north west) {Break Statements};
\end{tikzpicture}
%\bigskip
%---------------------------
\begin{tikzpicture}
\node [mybox] (box){%
    \begin{minipage}{0.3\textwidth}
While loops are the simplest loop in C. If condition is true then
the loop is executed over and over until it becomes false.
    \begin{lstlisting}
while ( condition ) {
// body of loop
}
    \end{lstlisting}
Do...while loops are like while loops, but the condition is evaluated
after the body so the body is always executed at least once.
    \begin{lstlisting}
do {
// body of loop
} while ( condition );
    \end{lstlisting}
A "For" Loop is used to repeat a specific block of code a known number of times.
\begin{lstlisting}
for ( start_statement ; condition ; end_statement ) {
// body of loop
}
\end{lstlisting}
    \end{minipage}
};
%------------ potencial titulo  ---------------------
\node[fancytitle, right=10pt] at (box.north west) {Loops};
\end{tikzpicture}

%---------------------------
\begin{tikzpicture}
\node [mybox] (box){%
    \begin{minipage}{0.3\textwidth}
    Assignments are expressions, so we can 
    get a condition to do
    double duty!
    \begin{lstlisting}
int i;
// Keep scanning as long as we get something
while ( r = scanf ("%d,%d",&i,&j) > 0) {
// process i , j , r
    }
    \end{lstlisting}
    \end{minipage}
};
%---------------------------------
\node[fancytitle, right=10pt] at (box.north west) {Assignments in conditions};
\end{tikzpicture}
%---------------------------
\begin{tikzpicture}
\node [mybox] (box){%
    \begin{minipage}{0.3\textwidth}
\begin{itemize}
\item In C, a string is \textbf{array of char}
\item "A" is a \textbf{string} constant
\item 'A' is a \textbf{char} constant
\item Strings end with a \textbf{NULL} character
\item String length must be specified to ensure it does not read/write beyon the end of the memory allocated
\end{itemize}
    \end{minipage}
};
%---------------------------------
\node[fancytitle, right=10pt] at (box.north west) {Strings};
\end{tikzpicture}

%\bigskip\\
%---------------------------
\begin{tikzpicture}
\node [mybox] (box){%
    \begin{minipage}{0.3\textwidth}
Arrays can be used to store:\\
- Numbers in a Vector\\
- Records in a databse\\
- Characters in a string\\
  Initialisation:
\begin{lstlisting}
// initialise with list , size = length of list
int a [] = { 1 , 2 , 3 };
// initialise with list , size explicit
// If list is to short , remaining elements set to 0
int b [4] = { 1 , 2 , 3 };
// Initialise with a loop
int c [4];
for ( int i = 0; i < 4; i ++) {
c [ i ] = -1;
}
\end{lstlisting}
We can also do multi-dimensional arrays which use more than one
index:
\begin{lstlisting}
// Store a matrix in a 2 d array
double matrix [3][3] = { { 1 , 0 , 0} ,
{ 0 , 1 , 0} ,
{ 0 , 0 , 1} };
// Store a b & w image
unsigned char picture [1920][1020];
\end{lstlisting}
\textbf{Array variables are just pointers!}
\begin{lstlisting}
- int * pX = &x
in English means an integer pointer named 'pX' is set to the address of x
- int A[10]; int* p = A; p[0] = 0; makes variable p point to the first member of array A.
\end{lstlisting}
    \end{minipage}
};
%---------------------------------
\node[fancytitle, right=10pt] at (box.north west) {Arrays};
\end{tikzpicture}

%\bigskip\\
%---------------------------
\begin{tikzpicture}
\node [mybox] (box){%
    \begin{minipage}{0.3\textwidth}
\nc{Pointer Definition}
A pointer is a variable that \textbf{stores the memory address of another variable as its value}. A pointer is created with the * operator.\\
    \end{minipage}
};
%---------------------------------
\node[fancytitle, right=10pt] at (box.north west) {Pointers};
\end{tikzpicture}
%---------------------------------
\begin{tikzpicture}
\node [mybox] (box){%
    \begin{minipage}{0.3\textwidth}
	\begin{lstlisting}
// We must allocate some space for a string
char buffer [100];
// scan in a string until the next whitespace
// Note : buffer is already a pointer ! No &
scanf ( " % s " , buffer );
// Scanf will happily write beyond the end of buffer .
	\end{lstlisting}
	\end{minipage}
};
%---------------------------------
\node[fancytitle, right=10pt] at (box.north west) {scanf() and strings};
\end{tikzpicture}
%---------------------------------
\begin{tikzpicture}
\node [mybox] (box){%
    \begin{minipage}{0.3\textwidth}
	\begin{lstlisting}
// Use %s to print a string with printf ()
char * mystring = " Hello world ! " ;
printf ( " mystring = % s \ n " , mystring );
	\end{lstlisting}
	\begin{lstlisting}
// Using a string variable as a format string :
char * myformat = " mystring = % s " ;
printf ( myformat , mystring );
	\end{lstlisting}
	\begin{lstlisting}
// puts () prints the string followed by newline
puts ( " Mystring = " );
puts ( mystring );
    \end{lstlisting}
	\end{minipage}
};
%---------------------------------
\node[fancytitle, right=10pt] at (box.north west) {Printing strings};
\end{tikzpicture}
%\bigskip\\
%---------------------------
\begin{tikzpicture}
\node [mybox] (box){%
    \begin{minipage}{0.3\textwidth}
The rand() function returns pseudorandom numbers between 0
and RAND\_MAX:\\
rand() is called only once using a specified seed.\\
If no \textbf{seed} is specified then it will always be the \textbf{default seed}.
\begin{lstlisting}
int r = rand (); // between 0 and RAND_MAX
int N = 100;
int s = rand () % N // 0 <= s < N
\end{lstlisting}
By default rand() will always return the same sequence
of numbers!. This is why we provide different seeds using srand()
    \end{minipage}
};
%---------------------------------
\node[fancytitle, right=10pt] at (box.north west) {rand()};
\end{tikzpicture}
%---------------------------
\begin{tikzpicture}
\node [mybox] (box){%
    \begin{minipage}{0.3\textwidth}
    \begin{itemize}
\item Provide seed using srand() before calling rand()
\item If the seed changes on each run then we get a new sequence
of numbers
\item Common to use current time as a seed
    \end{itemize}
    \end{minipage}
};
%---------------------------------
\node[fancytitle, right=10pt] at (box.north west) {srand()};
\end{tikzpicture}
%---------------------------
\bigskip\\
\begin{tikzpicture}
\node [mybox] (box){%
    \begin{minipage}{0.3\textwidth}
Functions allow reusing same code in many places and on different data
\begin{lstlisting}
int myfunction (int a , int b) {
return a + b ;
}
// Using a function
int x = myfunction (3 , 5);
// x = 8
// myfunction can now be called anywhere in the code
\end{lstlisting}
Functions \textsl{can} be passed like variables using pointers
\begin{lstlisting}
int addInt(int n, int m) {
    return n+m;
}
//define a pointer to a function which receives 2 ints and returns an int:
int (*functionPtr)(int,int);

//use as value
int sum = (*functionPtr)(2, 3); // sum == 5

//use as function
int add2to3(int (*functionPtr)(int, int)) {
    return (*functionPtr)(2, 3);
}
\end{lstlisting}
    \end{minipage}
};
%---------------------------------
\node[fancytitle, right=10pt] at (box.north west) {Functions};
\end{tikzpicture}
\begin{tikzpicture}
\node [mybox] (box){%
    \begin{minipage}{0.3\textwidth}
    \begin{tabularx}{1.0\textwidth} { 
  | >{\raggedright\arraybackslash}X 
  | >{\centering\arraybackslash}X | }
     \hline
     \textbf{Pass By Value} &\textbf{Pass By Reference}\\
     \hline
     Makes a copy of the actual param& Address of the actual param passed to func\\
     \hline
     Changes made inside function \textbf{do not reflect} original value & Changes made inside function \textbf{reflect} original value\\
     \hline
     Function gets a copy of the actual content & Function acesses the original variable's content\\
    \hline
    \end{tabularx}
    \end{minipage}
};
%---------------------------------
\node[fancytitle, right=10pt] at (box.north west) {Pass by Reference Vs Value};
\end{tikzpicture}
%---------------------------
%---------------------------
\begin{tikzpicture}
\node [mybox] (box){%
    \begin{minipage}{0.3\textwidth}
- Bitwise operators work on base 10 numbers, just turn them into binary\\

    \end{minipage}
};
%---------------------------------
\node[fancytitle, right=10pt] at (box.north west) {Extras};
\end{tikzpicture}
\begin{tikzpicture}
\node [mybox] (box){%
    \begin{minipage}{0.3\textwidth}
    A stack is a \textbf{linear data structure} in which insertions and deletions are allowed only \textbf{at the end}, called the \textbf{top of the stack}
    \begin{lstlisting}
// Space to store items
char stack [256];
// Stack starts at highest address and grows down !
char * stack_pointer = stack + 255;
// Puts the value x at the top of the stack
void push ( char x ) {
* stack_pointer = x ;
stack_pointer - -;
}
// Removes the last value by the function push
char pop () {
stack_pointer ++;
return * stack_pointer ;
}
    \end{lstlisting}
    \end{minipage}
};
%---------------------------------
\node[fancytitle, right=10pt] at (box.north west) {The stack};
\end{tikzpicture}
%---------------------------
\begin{tikzpicture}
\node [mybox] (box){%
    \begin{minipage}{0.3\textwidth}
\addtolength{\itemsep}{-2pt}
    \begin{tabularx}{1\textwidth} { 
  | >{\raggedright\arraybackslash}X 
  | >{\centering\arraybackslash}X 
  | >{\raggedleft\arraybackslash}X | }
     \hline
     Operator & Name & Operation \\
     \hline
     \huge{\~{}} & bitwise NOT & 1 if 0, 0 if 1\\
     \hline
     \&= & bitwise AND & 1 if both are 1, otherwise 0\\
     \hline
     \textbar & bitwise OR & 0 if both 0, otherwise 1\\
    \hline
    \^{} & bitwise XOR & 0 if the same, 1 if different\\
    \hline
    $<<$ & Left shift & move all bits left, fill right with 0\\
    \hline
    $>>$ & Right shift & move all bits right, fill left with 0\\
    \end{tabularx}
OR and AND for any bit b
\begin{lstlisting}
b | 1 == 1
b | 0 == b
b & 1 == b
b & 0 == 0

\end{lstlisting}
\end{minipage}
};
%---------------------------------
\node[fancytitle, right=10pt] at (box.north west) {Bitwise Operations};
\end{tikzpicture}
%---------------------------

\begin{tikzpicture}
\node [mybox] (box){%
    \begin{minipage}{0.3\textwidth}
    \begin{lstlisting}
Setting Bits:
x = 0b1010 ;
mask = 0b0011 ;
z = x | mask ; 
// z == 0b1011

Clearing Bits:
x = 0b1010 ;
mask = 0b0011 ;
z = x ^ mask ; 
// z == 0b1001

Testing Bits
x = 0b1010 ;
mask = 0b0011 ;
z = x & mask ; 
// z == 0 b0010
    \end{lstlisting}
\end{minipage}
};
%---------------------------------
\node[fancytitle, right=10pt] at (box.north west) {Bitwise Masks};
\end{tikzpicture}
%---------------------------
\begin{tikzpicture}
\node [mybox] (box){%
    \begin{minipage}{0.3\textwidth}
    \nc{Registers}
Storage with \textbf{8 bits capacity}\\
- Connected to \textbf{CPU} (accumulator)\\
- Operations on their content require \textbf{only one} instruction\\
\nc{Data Direction Registers (DDRx)}
Configured to specifiy which of the 8 bits is used for \textbf{output(1)} or \textbf{input(0)}\\
\nc{Data Register (PORTx)}
Writes output data to port\\
\nc{Input Pins Address (PINx)}
Reads input data from port
\begin{lstlisting}
unsigned char temp; // temporary variable
temp = PINB; // read input
\end{lstlisting}
\end{minipage}
};
%---------------------------------
\bigskip
\node[fancytitle, right=10pt] at (box.north west) {Micro-Controllers};
\end{tikzpicture}
\begin{tikzpicture}
\node [mybox] (box){%
    \begin{minipage}{0.3\textwidth}
\nc{Serial Vs Parallel}
\begin{tabularx}{1.0\textwidth} { 
  | >{\raggedright\arraybackslash}X 
  | >{\centering\arraybackslash}X | }
     \hline
     Serial & Parallel \\
     \hline
     One data bit is transceived at a time & Multiple data bits are transceived at a time\\
     \hline
     Slower&Faster\\
     \hline
     Less number of cables required&Higher number of cables required\\
    \end{tabularx}
    \nc{Synchronous Vs Asynchronous}
\begin{tabularx}{1.0\textwidth} { 
  | >{\raggedright\arraybackslash}X 
  | >{\centering\arraybackslash}X | }
     \hline
     Synchronous & Asynchronous\\
     \hline
     Send \& Receiver clocks synchronised & Not synchronised\\
     \hline
     Faster&Bytes are enclosed between start and stop bits\\
     \hline
     Example: Serial Peripheral Interface (SPI)& Example: Universal Sync/\textbf{Async} Receiver/Transmitter (USART)\\
    \end{tabularx}
    \nc{USART}
1) \textbf{Asynchronous Normal Mode}\\
--- Data is transferred at the BAUD rate set in UBBR register\\
--- Data is transmitted/received asynchronously, not using clock pulses.\\
2) \textbf{Asynchronous Double Speed Mode}\\
--- Everything is like normal mode but doubled\\
3) \textbf{Synchronous Mode}\\
--- Requires both data and a clock\\
--- The data is transmitted at a fixed rate\\
\nc{To use must specify}
1) Baud Rate\\
2) Number of data bits encoding a frame or char\\
--- Usually 10: 1 start, 8 data, 1 stop.\\
3) The sense of the parity bit\\
4) Number of stop bits\\
Minimally you'll need:
\begin{itemize}
\item uart\_{}init(ubrr) //Initialise the baud rate
\item uart\_{}putchar(unsigned char data) 
\item unsigned char c = uart\_{}getchar(void) 
\end{itemize}
\end{minipage}
};
%---------------------------------
\node[fancytitle, right=10pt] at (box.north west) {Serial Communications};
\end{tikzpicture}
%---------------------------------
\begin{tikzpicture}
\node [mybox] (box){%
    \begin{minipage}{0.3\textwidth}
\begin{itemize}
    \item AVR timers run asynchronous to main AVR core
    \item Use clock pulse as parallel increment
    \item When timer reaches TOP value, reset to 0 (overflowed)
    \item When overflowed, sends signal which can be used for interrupts
\end{itemize}
\begin{tabularx}{1.0\textwidth} { 
  | >{\raggedright\arraybackslash}X 
  | >{\centering\arraybackslash}X | }
     \hline
     \textbf{Pin} &\textbf{Output Compare Register}\\
     \hline
    \textbf{Pin 3} & \textbf{OC2B}\\
     \hline
    \textbf{Pin 5} & \textbf{OC0B}\\
     \hline
    \textbf{Pin 6} & \textbf{OC0A}\\
    \hline
    \textbf{Pin 9} & \textbf{OC1A}\\
    \hline
    \textbf{Pin 10} & \textbf{OC1B}\\
    \hline
    \textbf{Pin 11} & \textbf{OC2A}\\
    \hline
    \end{tabularx}
\end{minipage}
};
%---------------------------------
\node[fancytitle, right=10pt] at (box.north west) {Timers};
\end{tikzpicture}
%---------------------------------
\begin{tikzpicture}
\node [mybox] (box){%
    \begin{minipage}{0.3\textwidth}
\textbf{ADC Definition:}\\ 
Analog to Digital converters are used to create an \textbf{electric quantity}(Bit value) to vary directly like a \textbf{continuous variable} (temperature)
\nc{Registers Involved}
\begin{enumerate}
    \item \textbf{ADMUX} - used to select reference voltage source
    \item \textbf{ADCSRx} - used to tune aspects of the conversion(Auto-Trigger, Off/On,  Interrupt Enable)
\end{enumerate}
\end{minipage}
};
%---------------------------------
\node[fancytitle, right=10pt] at (box.north west) {ADC};
\end{tikzpicture}
%---------------------------------
\begin{tikzpicture}
\node [mybox] (box){%
    \begin{minipage}{0.3\textwidth}
    \nc{PWM Definition}
It is a way of simulating an \textbf{analog signal} via a \textbf{digital pin}. It does this buy turning \textbf{off} and \textbf{on} at different \textbf{frequencies} to achieve desired effects
\nc{Registers Involved}
\begin{enumerate}
    \item \textbf{TCCRx} - used to determine when to turn off and on
    \item \textbf{OCRxx} - Output Compare Register
\end{enumerate}
\end{minipage}
};
%---------------------------------
\node[fancytitle, right=10pt] at (box.north west) {PWM};
\end{tikzpicture}

\end{multicols*}
\end{document}


